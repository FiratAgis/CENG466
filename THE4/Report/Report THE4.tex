\documentclass[conference]{IEEEtran}
\IEEEoverridecommandlockouts
% The preceding line is only needed to identify funding in the first footnote. If that is unneeded, please comment it out.
\usepackage{cite}
\usepackage{amsmath,amssymb,amsfonts}
\usepackage{algorithmic}
\usepackage{graphicx}
\usepackage{textcomp}
\usepackage{xcolor}
\usepackage{float}
\def\BibTeX{{\rm B\kern-.05em{\sc i\kern-.025em b}\kern-.08em
    T\kern-.1667em\lower.7ex\hbox{E}\kern-.125emX}}
\begin{document}

\title{CENG466, Fall 2022, THE 3\\

}

\author{\IEEEauthorblockN{1\textsuperscript{st} Fırat Ağış}
\IEEEauthorblockA{\textit{Department of Computer Engineering} \\
\textit{Middle East Technical University}\\
Ankara, Turkey \\
e2236867@ceng.metu.edu.tr}
\and
\IEEEauthorblockN{2\textsuperscript{nd} Robin Koç}
\IEEEauthorblockA{\textit{Department of Computer Engineering} \\
\textit{Middle East Technical University}\\
Ankara, Turkey \\
e2468718@ceng.metu.edu.tr}
}

\maketitle

\begin{abstract}

\end{abstract}

\begin{IEEEkeywords}

\end{IEEEkeywords}



\section{Dependencies}
We used following libraries for the described reasons.
\begin{itemize}
	\item \textbf{os:} Handling non-existent input or output paths.
	\item \textbf{math:} Performing square root operation.
	\item \textbf{numpy:} Executing array and matrix operations.
	\item \textbf{PIL:} Reading images and converting them to arrays.
	\item \textbf{mathplotlib:} Creating histograms as graphics and writing arrays as image files.
\end{itemize}

\end{document}
